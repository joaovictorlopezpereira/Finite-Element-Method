\maketitle
\newpage

\begin{center}
  \textbf{Resumo:}
\end{center}

Este documento apresenta a resolução de um sistema genérico de equações diferenciais ordinárias, com uma componente não linear, utilizando o Método dos Elementos Finitos, conforme abordado nas aulas da disciplina \textit{Introdução ao Método dos Elementos Finitos}, ministradas pelo Marcello Goulart Teixeira e Bruno Alves do Carmo na Universidade Federal do Rio de Janeiro, durante o segundo semestre de 2024. Neste documento, apresentamos a resolução de uma equação do calor em uma dimensão espacial, com dependência temporal e não linearidade.

\vspace{0.3cm}

\begin{center}
  \textbf{Abstract:}
\end{center}

This document presents the solution of a generic system of ordinary differential equations, with a non-linear component, using the Finite Elements Method, as covered in the course \textit{Introduction to the Finite Element Method}, taught by professors Marcello Goulart Teixeira and Bruno Alves do Carmo at the Federal University of Rio de Janeiro, during the second half of 2024. In this document, we present the solution of a one-dimensional heat equation, with temporal dependence and non-linearity.

\vspace{0.3cm}

\begin{center}
  \textbf{Agradecimentos:}
\end{center}

  Agradeço ao professores Marcello Teixeira e Bruno Carmo e aos colegas Leonardo, Hashimoto e a vários outros que me ajudaram no entendimento do conteúdo necessário para a realização das contas e do Método dos Elementos Finitos como um todo.

\vspace{0.3cm}

\begin{center}
  \textbf{Thanks:}
\end{center}

  I would like to thank professors Marcello Teixeira and Bruno Carmo and colleagues Leonardo, Hashimoto and several other who helped me understand the necessary content for carrying out the calculations and the Finite Element Method as a whole.

\newpage
\tableofcontents
\newpage
