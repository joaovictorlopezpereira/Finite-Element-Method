% Language and Document
\usepackage[brazil]{babel}                % Document in Brazilian Portuguese
\usepackage[utf8]{inputenc}               % Listings recognize UTF-8 characters
\usepackage[T1]{fontenc}                  % Document allows proper copy-pasting

% Images
\usepackage{graphicx}                     % Image insertion
\usepackage{float}                        % Force exact image positioning [H]

% References and Figures
\usepackage{tocloft}                      % Table of Contents
\usepackage{hyperref}                     % Hyperlinks
\usepackage{csquotes}                     % Formatting of bibliographic citations
\usepackage[backend=biber]{biblatex}      % Bibliographic references
\addbibresource{repertoire.bib}           % Reference file
\graphicspath{{figures/}}                 % Image source directory used in the document

% Formatting
\usepackage{cancel}
\usepackage[top=1in, bottom=1in, left=0.5in, right=0.5in]{geometry}
\usepackage{amsmath, amssymb}             % Mathematical symbols
\usepackage{fancyvrb}                     % Verbatim commands
\usepackage{upquote}                      % Correct single quote in verbatim
\usepackage{indentfirst}                  % Indentation in the first paragraph
\usepackage{listings}                     % Insert code snippets (e.g., LISP)
\setlength{\parskip}{\baselineskip}       % Vertical spacing between paragraphs
\setlength{\footnotesep}{0.8em}           % Spacing between text and footer
\setlength{\skip\footins}{2em}            % Spacing before the footer
\setlength{\cftbeforechapskip}{12pt}      % Spacing before chapters in TOC
\setlength{\cftbeforesecskip}{6pt}        % Spacing before sections in TOC
\setlength{\jot}{10pt}

% Title
\title{
  Resolução do Problema Unidimensional \\
  Estacionário com Condição de \\
  fronteira de Dirichlet Homogênea \\
  via o Método dos Elementos Finitos
}

% Date
\date{
  \today\\
  \vfill
  Rio de Janeiro - RJ
}

% Authors
\author{
  João Victor Lopez Pereira
}

% Code Environment
\lstnewenvironment{code}[1][]{             % Scheme Code
  \lstset{
    basicstyle=\ttfamily,
    columns=flexible,                      % Testing as flexible
    breaklines=true,
    breakatwhitespace=true,
    frame=none,
    basewidth=0.5em,                       % Ensures fixed-width among letters
    aboveskip=13pt,
    belowskip=0pt,
    #1
  }
}{}

% Multiline Code Output
\lstnewenvironment{styledoutput}[1][]{
  \lstset{
    basicstyle=\ttfamily\itshape,
    breaklines=true,
    breakatwhitespace=true,
    frame=none,
    aboveskip=0pt,
    #1
  }
}{}

% Code Input
\newcommand{\userinput}[1]{
  \noindent
  \texttt{\textit{>}}
  \texttt{#1}
}

% Code Output
\newcommand{\useroutput}[1]{              % REPL output to the user
  \textit{\texttt{#1}}
}

% Bibliographic Citation
\newcommand{\excerpt}[4]{                 % {quote}{author}{source}{cite}
  \vspace{25pt}
  \begin{quote}
    \begin{center}
      \hfill
      \begin{minipage}{0.65\textwidth}
        ``#1''

        \vspace{5pt}---#2, \\
        \textit{``#3''}#4
      \end{minipage}
    \end{center}
  \end{quote}
  \vspace{25pt}
}

% Mathematical Function/Formula Definition
\newcommand{\mathformula}[1]{
  \vspace{7pt}
    \[#1\]
  \vspace{7pt}
}

% Variables
\newcommand{\var}[1]{\texttt{#1}}

% Foreign Words
\newcommand{\nonport}[1]{\textit{#1}}
